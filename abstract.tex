\part*{Abstract Algebra}

\section*{Integers}

\begin{fact}[The Division Algorithm]
	If $a$ and $n$ are integers and $n$ is positive, then there exists unique
	integers $q$ and $r$ such that $a = qn + r$ and $0 \leq r < n$.
\end{fact}

\begin{fact}
	If $\gcd(a, b) = d$, then there exists $m, n \in \Z$ such that $ma + nb = d$.
\end{fact}

\section*{Groups}

\begin{definition}
	Suppose that:
	\begin{enumerate}
		\item $G$ is a set and $*$ is a binary operation on $G$,
		\item $*$ is associative,
		\item there exists an element $e$ of $G$ such that $\forall x \in G$ \[x
			* e = e * x = x \text{ (identity element)}\]
		\item for each $x \in G$, there exists an element $y \in G$ such that
			\[x * y = y * x = e\]
	\end{enumerate}
	Then $G$, together with the binary operation $*$, is called a
	\emph{group} and denoted $(G, *)$.
\end{definition}

\begin{theorem}
	If $(G, *)$ is a group, then there is only one identity element in $G$.
\end{theorem}

\begin{theorem}
	If $(G, *)$ is a group and $x \in G$, then $x$ has only one inverse.
\end{theorem}

\begin{definition}
	We say a group $G$ is \emph{abelian} if the group operation is commutative,
	i.e., if $xy = yx$ for all $x, y \in G$.
\end{definition}

\begin{notation}
	For $g \in G$ and $n \in \Z$, $g^n = g \cdot g \cdots g$ ($n$-times).
\end{notation}

\begin{notation}
	If the operation of $G$ is called $+$, then we write $ng$ instead of $g^n$.
\end{notation}

\begin{notation}
	When $G$ is finite, its order $|G|$ is the number of elements in the group.
\end{notation}

\begin{notation}
	When $g \in G$ has finite order, its order $o(g)$ is the smallest integer $m >
	0$ with $g^m = e$.
\end{notation}

\begin{notation}
	If $g^m = e$ for some $m \in \Z$, then $o(g) \mid m$.
\end{notation}

\section*{Subgroups}

\begin{definition}
	Let $(G, *)$ be a group and $H \subseteq G$. $H$ is a \emph{subgroup} of $G$
	if the elements of $H$ from a group under $*$. I.e. $(H, *)$ is a group.
\end{definition}

\begin{theorem}
	Let $H$ be a nonempty subset of group $G$. Then, $H$ is a subgroup of $G$ if
	and only if
	\begin{enumerate}
		\item $\forall a,b \in H$, $ab \in H$ and
		\item $\forall a \in H$, $a^{-1} \in H$.
	\end{enumerate}
	\emph{Terminology:} If $H$ has property 1 we say it is closed under
	multiplication. If $H$ has property 2 we say it is closed under inverses.
\end{theorem}

\begin{definition}
	An element $g \in G$ generates the \emph{cyclic} subgroup $\langle g
	\rangle = \{g^m \mid m \in \Z\}$.
\end{definition}

\begin{theorem}
	An element $g \in G$ has finite order if and only if $\langle g \rangle$ is
	finite, in which case $o(g) = |\langle g \rangle|$.
\end{theorem}

\begin{theorem}[Lagrange's Theorem]
	If $H$ is a subgroup of a finite group $G$, then $|H| \mid |G|$.
\end{theorem}

\begin{corollary}
	If $|G|$ is prime, the $G$ is cyclic.
\end{corollary}

\begin{corollary}
	For all $g \in G$, we have $o(g) \mid |G|$.
\end{corollary}

\begin{corollary}
	For all $g \in G$, we have $g^{|G|} = e$.
\end{corollary}

\section*{Cosets}

\begin{definition}
	If $H$ is a subgroup of $G$, then by \emph{right coset} of $H$ in $G$ we
	mean a subset of the form $Hg$, where $g \in G$ and
	\[
		Hg = \{hg \mid h \in H\}.
	\]
\end{definition}

\begin{definition}
	If $H$ is a subgroup of $G$, then by \emph{left coset} of $H$ in $G$ we
	mean a subset of the form $gH$, where $g \in G$ and
	\[
		gH = \{gh \mid h \in H\}.
	\]
\end{definition}

\begin{theorem}
	Two right cosets $Hx$ and $Hy$ are either the same set or disjoint sets.
	(That is, if they share even one element, they are exactly the same set.) The
	same holds for left cosets. (On the other hand, a right coset and a left
	coset can intersect each other without being the same set.)
\end{theorem}

\begin{corollary}[Right coset relation]
	$Hx = Hy \Longleftrightarrow xy^{-1} \in H \Longleftrightarrow x \in Hy$.
\end{corollary}

\begin{corollary}[Left coset relation]
	$xH = yH \Longleftrightarrow y^{-1}x \in H \Longleftrightarrow x \in yH$.
\end{corollary}

\begin{corollary}
	$Hx = H \Longleftrightarrow x \in H \Longleftrightarrow xH = H$.
\end{corollary}

\begin{notation}
	If $G$ is abelian and the group operation is written as addition, then the
	left and right cosets of $H \subseteq G$ coincide and are written
	\[
		H + a = \{h + a \mid h \in H\}.
	\]
	Here, the coset relation becomes
	\[
		H + a = H + b \Longleftrightarrow a - b \in H \Longleftrightarrow a \in H +
		b.
	\]
\end{notation}

\begin{definition}
	When a group $G$ is a union of finitely many left cosets of a subgroup $H$, we
	say that $H$ has finite index in $G$ and the \emph{index} of $H$ in $G$ is
	defined to be
	\[
		[G : H] = \text{number of distinct left cosets of $H$ in $G$}.
	\]
	The same holds for right cosets. When $G$ is finite, $[G : H] = |G|/|H|$,
	since all cosets of $H$ have the same number of elements.
\end{definition}

\section*{Normal Subgroups}

\begin{theorem}
	Given a subgroup $H \subseteq G$, $H$ being normal in $G$ is equivalent to
	any of the following conditions:
	\begin{itemize}
		\item $gH = Hg$ for all $g \in G$.
		\item $gHg^{-1} = H$ for all $g \in G$.
		\item $ghg^{-1} \in H$ for all $g \in G$ and $h \in H$.
	\end{itemize}
\end{theorem}

\begin{theorem}
	When $N \subseteq G$ is a normal subgroup, every left coset is a right coset,
	and vice versa.  The set of all cosets of $N$ in $G$ forms a group and is
	denoted $G/N$. The group operation is defined by $Na \cdot Nb = Nab$, which is
	well-defined since $N$ is normal. When $G$ is finite, $G/N$ is also finite and
	$|G/N| = [G : N] = |G|/|N|$.
\end{theorem}

\section*{Group Homomorphisms}

\begin{definition}
	Let $G$ and $H$ be groups and let $\phi : G \rightarrow H$ be a function. We
	say that $\phi$ is a \emph{homomorphism} if for all $a, b \in G$,
	\[
		\phi(ab) = \phi(a) \phi(b).
	\]
\end{definition}

\begin{theorem}
	If $\phi : G \rightarrow H$ is a homomorphism, then
	\begin{itemize}
		\item $\phi(e_G) = e_H$.
		\item $\phi(g^n) = {\phi(g)}^n$ for all $g \in G$ and $n \in \Z$.
	\end{itemize}
\end{theorem}

\begin{definition}
	If $\phi : G \rightarrow H$ is a homomorphism, then
	\begin{itemize}
		\item The \emph{kernel} of $\phi$ is $\Ker(\phi) = \{g \in G \mid \phi(g) =
			e_H\} \subseteq G$.
		\item The \emph{image} of $\phi$ is $\Image(\phi) = \{\phi(g) \mid g \in
			G\} \subseteq H$.
	\end{itemize}
\end{definition}

\begin{theorem}
	If $\phi : G \rightarrow H$ is a homomorphism, then $\Ker(\phi)$ is a normal
	subgroup of $G$.
\end{theorem}

\begin{theorem}
	If $\phi : G \rightarrow H$ is a homomorphism, then $\Image(\phi)$ is a
	subgroup of $H$, but not necessarily normal.
\end{theorem}

\begin{theorem}
	Given $\phi : G \rightarrow H$ is a homomorphism, $\phi$ is one-to-one if and
	only if $\Ker(\phi) = \{e_G\}$.
\end{theorem}

\begin{theorem}
	Given a group homomorphism $\phi : G \rightarrow H$, $\phi$ being an
	isomorphism is equivalent to
	\begin{itemize}
		\item $\phi$ is one-to-one and onto.
		\item $\phi$ has an inverse function $\phi^{-1} : H \rightarrow G$ that is a
			group homomorphism.
	\end{itemize}
\end{theorem}

\begin{theorem}[The Fundamental Theorem of Group Homomorphisms]
	If $\phi : G \rightarrow H$ is a group homomorphism, then there is a group
	isomorphism $\widetilde{\phi} : G/\Ker(\phi) \simeq \Image(\phi)$ defined
	by $\widetilde{\phi}(g \Ker(\phi)) = \phi(g)$.
\end{theorem}

\section*{Permutations}

\begin{definition}
	If $X$ is a nonempty set, then a one-to-one onto function $f : X \rightarrow
	X$ is called a \emph{permutation}.
\end{definition}

\begin{definition}
	Let $X$ be a nonempty set. The group $(S_X, \circ)$ is called the
	\emph{symmetric group on $X$}. If $X$ is a finite set, there is no harm in
	assuming that $X = \{1, 2, \dots, n\}$. In this case, we denote the group
	$(S_X, \circ)$ by $S_n$.
\end{definition}

\begin{notation}
	If $\sigma \in S_n$, we can represent $\sigma$ by an array
	\[
		\begin{pmatrix}
			1 & 2 & 3 & \cdots & n \\
			\sigma(1) & \sigma(2) & \sigma(3) & \cdots & \sigma(n)
		\end{pmatrix}.
	\]
\end{notation}

\begin{fact}
	$|S_n| = n$!
\end{fact}

\begin{definition}
	Given distinct $i_1, \dots, i_n$, the $n$-cycle $(i_1 \ i_2 \cdots i_n)$ maps
	$i_1$ to $i_2$, $i_2$ to $i_3$, $\dots$, $i_n$ to $i_1$, and is the identity
	elsewhere.
\end{definition}

\begin{fact}
	The order of an $n$-cycle is its length $n$.
\end{fact}

\begin{definition}
	Two cycles $(i_1 \ i_2 \cdots i_n)$ and $(j_1 \ j_2 \cdots j_n)$ are
	\emph{disjoint} if
	\[
		\{i_1, \dots, i_n\} \cap \{j_1, \dots, j_n\} = \emptyset.
	\]
\end{definition}

\begin{fact}
	If $\sigma \in S_n$ is written as a product of disjoint cycles $\sigma =
	\sigma_1 \cdots \sigma_k$, then
	\[
		o(\sigma) = \lcm(\text{length } \sigma_1, \dots, \text{length } \sigma_k).
	\]
\end{fact}

\begin{definition}
	``Transposition'' is just another word for 2-cycle.
\end{definition}

\begin{fact}
	Every element of $S_n$ can be written as a product of transpositions. A given
	$\sigma \in S_n$ can be written as a product of transpositions in many ways,
	this product is \emph{not} a unique factorization.
\end{fact}

\begin{definition}
	A permutation is said to be \emph{even} if it can be written as the product
	of an even number of transpositions. It is \emph{odd} if it can be written as
	the product of an odd number of transpositions.
\end{definition}

\begin{definition}
	The alternating group $A_n$ consists of all even permutations in $S_n$.
\end{definition}

\begin{fact}
	For $n \geq 2$, $|A_n| = \frac{n!}{2}$.
\end{fact}

\begin{fact}
	An $n$-cycle can be written as a product of $n - 1$ transpositions. In
	particular, every cycle of odd length has odd order but is an even
	permutation. Similarly, every cycle of even length has even order but is an
	odd permutation.
\end{fact}

\section*{Rings}

\begin{definition}
	Suppose that $R$ is a set and $+$ and $\cdot$ are two binary operations on
	$R$. Further suppose that:
	\begin{enumerate}
		\item $(R, +)$ is an abelian group
		\item $\cdot$ is associative
		\item $\forall r_1, r_2, r_3 \in R$,
			\begin{align*}
				r_1 (r_2 + r_3) &= r_1 r_2 + r_1 r_3 \text{ and } \\
				(r_2 + r_3) r_1 &= r_2 r_1 + r_3 r_1.
			\end{align*}
	\end{enumerate}
	Then, $R$ together with $+$ and $\cdot$ is called a \emph{ring}. We denote it
	by $(R, +, \cdot)$ or just $R$ for short.
\end{definition}

\begin{definition}
	If $\cdot$ is commutative, then $R$ is called a \emph{commutative ring}.
\end{definition}

\begin{notation}
	The \emph{additive} identity element of $R$ is denoted by $0_R$ or just $0$.
\end{notation}

\begin{definition}
	If $\cdot$ has a \emph{multiplicative} identity element, it is unique and
	denoted $1_R$ or just $1$. We call $1_R$ the \emph{unity} of $R$ if it exists.
	A ring with unity is called a \emph{ring with unity}.
\end{definition}

\begin{definition}
	If $R$ is a ring with unity, then any $x \in R$ that has a
	\emph{multiplicative} inverse $x^{-1}$ is called a \emph{unit}. The set of all
	units of $R$ is denoted $R^\times$ and forms a group under the multiplication
	operation, with identity element $1_R$.
\end{definition}

\begin{definition}
	A ring $R$ is called a \emph{division ring} if $R$ has a unity $1 \neq 0$ and
	every nonzero element of $R$ is a unit. A commutative division ring is called
	a \emph{field}.
\end{definition}

\begin{facts}
	Suppose $R$ is a ring and $x \in R$.
	\begin{itemize}
		\item $0_R x = x 0_R = 0_R$.
		\item Since $(R, +)$ forms a group, for an integer $n$, we write $nx$ to
			denote $x$ added to itself $n$ times (or subtracted, if $n$ is negative;
			or $0_R$ if $n = 0$).
		\item For a \emph{positive} integer $n$, we write $x^n$ for $x$ multiplied
			by itself $n$ times. If $R$ has unity, then $x^0 = 1_R$; if in addition
			$x$ is a unit (i.e., if $x$ has multiplicative inverse), then $x^{-n} =
			{(x^{-1})}^n$.
	\end{itemize}
\end{facts}

\begin{definition}
	If $R$ is a commutative ring, the \emph{polynomial ring} $R[x]$ consists of
	all polynomials in $x$ with coefficients in $R$.
\end{definition}

\section*{Ideals}

\begin{definition}
	A subset $I \subseteq R$ is an \emph{ideal} if it satisfies the following
	properties:
	\begin{enumerate}
		\item $I \neq \emptyset$.
		\item For all $x, y \in I$, we have $x - y \in I$.
		\item For all $x \in I$ and $r \in R$, we have $rx \in I$ and $xr \in I$.
	\end{enumerate}
	Properties 1 and 2 say that $I$ is a subgroup under addition, and property 3
	is sometimes informally called the ``sticky'' property.
\end{definition}

\begin{facts}
	Some facts about ideals.
	\begin{itemize}
		\item When the ring is commutative, we have $rx = xr$, so the sticky
			property simplifies to: For all $x \in I$ and $r \in R$, we have $rx \in
			I$.
		\item An ideal $I$ always contains the zero element of the ring.
		\item Let $R$ be a ring with unity 1, and let $I \subseteq R$ be an ideal.
			Then $I$ contains 1 if and only if $I = R$.
	\end{itemize}
\end{facts}

\section*{Quotient Rings}

\begin{definition}
	Recall that an ideal $I \subseteq R$ is a group under addition, so the
	cosets of $I$ are usually written
	\[
		I + r = \{s + r \mid s \in I\},
	\]
	although sometimes you see $r + I$ since addition is commutative. The
	definition of ideal guarantees that the set of cosets
	\[
		R/I = \{I + r \mid r \in R\}
	\]
	becomes a ring, called the quotient ring, under the following
	operations:
	\[
		(I + a) + (I + b) = I + (a + b) \quad \text{and} \quad (I + a) (I + b) = I +
		ab.
	\]
\end{definition}

\section*{Ring Homomorphisms}

\begin{definition}
	Let $R$ and $S$ be rings and let $\phi : R \rightarrow S$ be a function. Then,
	$\phi$ is a \emph{ring homomorphism} if for all $x, y \in R$:
	\begin{enumerate}
		\item $\phi(x + y) = \phi(x) + \phi(y)$.
		\item $\phi(xy) = \phi(x) \phi(y)$.
	\end{enumerate}
	We define \emph{isomorphism} just as before.
\end{definition}

\begin{facts}
	Some facts about ring homomorphisms.
	\begin{itemize}
		\item If $\phi : R \rightarrow s$ is a ring homomorphism, then $\phi(0_R) =
			0_S$. [But even if both rings have 1, we might \emph{not} have $\phi(1_R)
			= 1_S$!]
		\item If $\phi : R \rightarrow S$ is a ring homomorphism, then for all $x
			\in R$ and all $n \in \Z$, we have $\phi(nx) = n \phi(x)$.
		\item If $\phi : R \rightarrow S$ is a ring homomorphism, then for all $x
			\in R$ and all $n \in \N$, we have $\phi(x^n) = {\phi(x)}^n$.
	\end{itemize}
\end{facts}

\begin{theorem}
	If $\phi : R \rightarrow S$ is a ring homomorphism, then $\Ker(\phi)$ is an
	ideal of $R$.
\end{theorem}

\begin{theorem}[The Fundamental Theorem of Ring Homomorphisms]
	If $\phi : R \rightarrow S$ is a ring homomorphism, then there is a ring
	homomorphism $\widetilde{\phi} : R/\Ker(\phi) \simeq \Image(\phi)$ defined by
	$\widetilde{\phi}(\Ker(\phi) + r) = \phi(r)$.
\end{theorem}

\section*{Quotient Rings and Fields}

\begin{theorem}
	A commutative ring with unity is a field if and only if its only ideals are
	$\{0\}$ and the whole ring.
\end{theorem}

\begin{definition}
	An ideal $I$ of a ring $R$ is called a \emph{maximal ideal} if $I$ is a proper
	ideal and there is no other proper ideal $J$ such that $I \subsetneq J$.
\end{definition}

\begin{theorem}
	Let $R$ be a commutative ring with unity. Ideal $M \subseteq R$ is maximal if
	and only if $R/M$ is a field.
\end{theorem}

\section*{Polynomial Rings $k[x]$, for a Field $k$}

\begin{definition}
	A polynomial $f = f(x) \in k[x]$ is a symbolic object ($x$ is just a symbol)
	in the ring $k[x]$. However, if we replace $x$ with an element $a \in k$
	(sometimes called ``plugging in''), then we get an element $f(a) \in k$. This
	operation is compatible with addition and multiplication.
\end{definition}

\begin{theorem}[The Division Algorithm]
	Let $k$ be a field and let $f(x), g(x) \in k[x]$. If $g(x) \neq 0$, then there
	exists $q(x), r(x) \in k[x]$ such that
	\[
		f(x) = q(x) g(x) + r(x)
	\]
	and either $r(x) = 0$ or $\deg(r(x)) < \deg(g(x))$.
\end{theorem}

\begin{definition}
	Let $R$ be a commutative ring with unity and let $a \in R$. Define
	\[
		aR = \{ar \mid r \in R\}
	\]
	Then, $aR$ is an ideal of $R$ $aR$ is called the \emph{principal ideal
	generated by $a$}.
\end{definition}
