\part*{Multivariable Calculus}

\section*{Elementary Vector Analysis}

\begin{notation}
	The Cartesian coordinates: $(x, y) \in \R^2$ and $(x, y, z) \in \R^3$.
\end{notation}

\begin{notation}
	A vector $\vec{v}$ in $\R^2$ or $\R^3$ is often represented by a directed line
	segment. In terms of coordinates, we write $\vec{v} = \langle a_1, a_2
	\rangle$ in $\R^2$ and $\vec{v} = \langle a_1, a_2, a_2 \rangle$ in $\R^3$.
\end{notation}

\begin{notation}
	The standard basis vectors $\vec{\imath} = \langle 1, 0 \rangle$,
	$\vec{\jmath} = \langle 1, 0 \rangle$ in $\R^2$ and $\vec{\imath} = \langle 1,
	0, 0 \rangle$, $\vec{\jmath} = \langle 0, 1, 0 \rangle$, $\vec{k} = \langle 0,
	0, 1 \rangle$ in $\R^3$.
\end{notation}

\begin{notation}
	A vector $\vec{v}$ has length $|\vec{v}|$, sometimes denoted $||\vec{v}||$.
\end{notation}

\begin{fact}
	Nonzero vectors $\vec{v}$ and $\vec{u}$ are parallel if and only if each is a
	constant multiple of the other.
\end{fact}

\begin{definition}
	A point $P$ in $\R^2$ or $\R^3$ gives a vector from the origin to $P$, called
	the position vector of $P$. This allows us to regard points as vectors and
	vice versa.
\end{definition}

\begin{definition}
	The length of a $n$-dimensional vector $\vec{v} = \langle a_1, \dots, a_n
	\rangle$ is
	\[
		|\vec{v}| = \sqrt{{a_1}^2 + \cdots + {a_n}^2}
	\]
\end{definition}

\begin{definition}
	In $\R^2$, the dot (scalar) product of $\vec{u} = \langle a_1, a_2 \rangle$
	and $\vec{v} = \langle b_1, b_2 \rangle$ is $\vec{u} \cdot \vec{v} = a_1 b_1 +
	a_2 b_2$, and similarly in $\R^3$, the dot product of $\vec{u} = \langle a_1,
	a_2, a_3, \rangle$ and $\vec{v} = \langle b_1, b_2, b_3 \rangle$ is $\vec{u}
	\cdot \vec{v} = a_1 b_1 + a_2 b_2 + a_3 b_3$.
\end{definition}

\begin{facts}
	The linearity properties of the dot product:
	\begin{enumerate}
		\item $\vec{u} \cdot \vec{u} = {|\vec{u}|}^2$.
		\item $\vec{u} \cdot \vec{v} = \vec{v} \cdot \vec{u}$.
		\item $\vec{u} \cdot (\vec{v} + \vec{w}) = \vec{u} \cdot \vec{v} + \vec{u}
			\cdot \vec{w}$.
		\item $(c \vec{u}) \cdot \vec{v} = c(\vec{u} \cdot \vec{v}) = \vec{v} \cdot
			(c \vec{v})$.
		\item $\vec{0} \cdot \vec{u} = 0$.
	\end{enumerate}
	Additional properties:
	\begin{enumerate}
		\item $\vec{u} \cdot \vec{v} = |\vec{u}| |\vec{v}| \cos \theta$, where
			$\theta$ is the angle between $\vec{u}$ and $\vec{v}$.
		\item $\vec{u} \cdot \vec{v} = 0$ if and only if $\vec{u}$ and $\vec{v}$ are
			perpendicular (orthogonal).
	\end{enumerate}
\end{facts}

\begin{definition}
	Given $\vec{u} = \langle a_1, a_2, a_3 \rangle$ and $\vec{v} = \langle b_1,
	b_2, b_3 \rangle$ in $\R^3$, their cross (vector) product is
	\[
		\vec{u} \times \vec{v} = \det
		\begin{pmatrix}
			\vec{\imath} & \vec{\jmath} & \vec{k} \\
			a_1 & a_2 & a_3 \\
			b_1 & b_2 & b_3
		\end{pmatrix}
		= \det
		\begin{pmatrix}
			a_2 & a_3 \\
			b_2 & b_3
		\end{pmatrix}
		\vec{\imath} -
		\det
		\begin{pmatrix}
			a_1 & a_3 \\
			b_1 & b_3
		\end{pmatrix}
		\vec{\jmath} +
		\det
		\begin{pmatrix}
			a_1 & a_2 \\
			b_1 & b_2
		\end{pmatrix}
		\vec{k}.
	\]
\end{definition}

\begin{facts}
	The linearity properties of the cross product:
	\begin{enumerate}
		\item $\vec{u} \times \vec{v} = -\vec{v} \times \vec{u}$.
		\item $(c \vec{u}) \times \vec{v} = c(\vec{u} \times \vec{v}) = \vec{u}
			\times (c \vec{v})$.
		\item $\vec{u} \times (\vec{v} + \vec{w}) = \vec{u} \times \vec{v} + \vec{u}
			\times \vec{w}$.
		\item $(\vec{u} + \vec{v}) \times \vec{w} = \vec{u} \times \vec{w} + \vec{v}
			\times \vec{w}$.
		\item $\vec{u} \cdot (\vec{v} \times \vec{w}) = (\vec{u} \times \vec{v})
			\cdot \vec{w}$.
		\item $\vec{u} \times (\vec{v} \times \vec{w}) = (\vec{u} \cdot \vec{w})
			\vec{v} - (\vec{u} \cdot \vec{v}) \vec{w}$.
	\end{enumerate}
	Additional properties:
	\begin{enumerate}
		\item $|\vec{u} \times \vec{v}| = |\vec{u}| |\vec{v}| \sin \theta$, where
			$\theta$ is the angle between $\vec{u}$ and $\vec{v}$.
		\item $\vec{u} \times \vec{v} = \vec{0}$ if and only if $\vec{u}$ and
			$\vec{v}$ are parallel.
		\item $\vec{u} \times \vec{v}$ is perpendicular to both $\vec{u}$ and
			$\vec{v}$.
	\end{enumerate}
\end{facts}

\begin{definition}
	In $\R^2$ or $\R^3$, a point $\vec{r}_0$ and a nonzero vector $\vec{v}$
	determine the line parametrized by
	\[
		\vec{r}(t) = \vec{r}_0 + t \vec{v}.
	\]
	The vector $\vec{v}$ is called a direction vector of the line. The parametric
	equations of a line for the coordinates $(x, y, z) \in \R^3$ are
	\[
		x = x_0 + at \quad y = y_0 + bt \quad z = z_0 + ct,
	\]
	where $\vec{v} = \langle a, b, c \rangle$, $\vec{r}_0 = \langle x_0, y_0, z_0
	\rangle$, and $t \in \R$. For $\R^2$, omit $z$.
\end{definition}

\begin{definition}
	A plan in $\R^3$ is defined by an equation of the form $ax + by + cz = d$
	where $\langle a, b, c \rangle \neq \langle 0, 0, 0 \rangle$. A more geometric
	way to write the equation uses a nonzero vector $\vec{n}$ perpendicular to the
	plane and point $(x_0, y_0, z_0)$ in the plan. Then:
	\begin{align*}
		(x, y, z) \text{ is in the plane } &\Longleftrightarrow \vec{n} \text{ is
		perpendicular to the vector from } (x, y, z) \text{ to } (x_0, y_0, z_0) \\
		&\Longleftrightarrow \vec{n} \cdot \langle x - x_0, y - y_0, z - z_0 \rangle
		= 0.
	\end{align*}
	The vector $\vec{n}$ is called a normal vector to the plane. For a plane
	defined by $ax + by +cz = d$, a normal vector is given by $\vec{n} = \langle
	a, b, c \rangle$.
\end{definition}

\begin{definition}
	Given a curve parameterization $\vec{r}(t) = (x(t), y(t))$ in the plane, the
	tangent vector to the curve at the point $\vec{r}(t)$ is
	\[
		\vec{r^\prime}(t) = \langle x^\prime(t), y^\prime(t) \rangle.
	\]
	The situation is similar on $\R^3$.
\end{definition}

\section*{Functions of Several Variables}

\begin{definition}
	If $f$ is a function of two variables, its partial derivatives are the
	functions $f_x$ and $f_y$ defined by
	\begin{align*}
		f_x(x, y) &= \lim_{h \rightarrow 0} \frac{f(x + h, y) - f(x, y)}{h} \\
		f_y(x, y) &= \lim_{h \rightarrow 0} \frac{f(x, y + h) - f(x, y)}{h}
	\end{align*}
	If $f$ is a function of three variables $x$, $y$, and $z$, then its partial
	derivative with respect to $x$ is defined as
	\[
		f_x(x, y, z) = \lim_{h \rightarrow 0} \frac{f(x +h, y, z) - f(x, y, z)}{h}
	\]
\end{definition}

\begin{notation}
	The standard notation for the partial derivatives: $\frac{\partial f}{\partial
	x} = f_x(x, y)$, $\frac{\partial f}{\partial y} = f_y(x, y)$,
	$\frac{\partial^2 f}{\partial^2 x} = f_{xx}(x, y)$, $\frac{\partial^2
	f}{\partial x \partial y} = f_{yx}(x, y)$, $\frac{\partial^2 f}{\partial^2 y}
	= f_{yy}(x, y)$ for $f(x, y)$, and similarly for $f(x, y, z)$.
\end{notation}

\begin{interpretation}
	If $z = f(x, y)$, then $\partial z / \partial x$ represents the rate of change
	of $z$ with respect to $x$ when $y$ is fixed. Similarly, $\partial z /
	\partial y$ represents the rate of change of $z$ with respect to $y$ when $x$
	is fixed.

	If $w = f(x, y, z)$, then $f_x = \partial w / \partial x$ can be interpreted
	as the rate of change of $w$ with respect to $x$ when $y$ and $z$ are held
	fixed.
\end{interpretation}

\begin{fact}
	Rule for find partial derivatives of $z = f(x, y)$.
	\begin{enumerate}
		\item To find $f_x$, regard $y$ as a constant and differentiate $f(x, y)$
			with respect to $x$.
		\item To find $f_y$, regard $x$ as a constant and differentiate $f(x, y)$
			with respect to $y$.
	\end{enumerate}
\end{fact}

\begin{definition}
	A unit vector is a vector whose length is 1. In general, if $\vec{v} \neq
	\vec{0}$, then the unit vector that has the same direction as $\vec{v}$ is
	\[
		\vec{u} = \frac{\vec{v}}{|\vec{v}|}.
	\]
\end{definition}

\begin{definition}
	The directional derivative of $f(x, y)$ in the direction of a unit vector
	$\vec{u} = \langle u_1, u_2 \rangle$ at the point $(a, b)$ is
	\[
		D_{\vec{u}} f(a, b) = \lim_{h \rightarrow 0} \frac{f(a + h u_1, b + h u_2) -
		f(a, b)}{h}
	\]
	if this limit exists.

	The directional derivative of $f(x, y, z)$ in the direction of a unit vector
	$\vec{u} = \langle u_1, u_2, u_3 \rangle$ at the point $(a, b, c)$ is
	\[
		D_{\vec{u}} f(a, b, c) = \lim_{h \rightarrow 0} \frac{f(a + h u_1, b + h
		u_2, c + h u_3) - f(a, b, c)}{h}
	\]
	if this limit exists.
\end{definition}

\begin{interpretation}
	The directional derivative is the rate of change of a function of two or more
	variables in any direction.
\end{interpretation}

\begin{definition}
	The gradient of $f(x, y)$ at $(a, b)$ is the vector
	\[
		\nabla f(a, b) = \frac{\partial f}{\partial x}(a, b) \vec{\imath} +
		\frac{\partial f}{\partial y}(a, b) \vec{\jmath} = \left\langle
		\frac{\partial f}{\partial x}(a, b), \frac{\partial f}{\partial y}(a, b)
		\right\rangle
	\]
	and similarly for $f(x, y, z)$.
\end{definition}

\begin{fact}
	$\nabla f(a, b)$ is perpendicular to the level curve $f(x, y) = f(a, b)$ at
	the point $(a, b)$. Similarly, $\nabla f(a, b, c)$ is perpendicular to the
	level surface $f(x, y, z) = f(a, b, c)$ at $(a, b, c)$.
\end{fact}

\begin{theorem}
	If $f(a, b)$ is differentiable at $(a, b)$ and $\vec{u}$ is a unit vector,
	then
	\[
		D_{\vec{u}} f(a, b) = \nabla f(a, b) \cdot \vec{u},
	\]
	and similarly for $f(x, y, z)$.
\end{theorem}

\begin{theorem}
	When $\nabla f(a, b) \neq \vec{0}$, the unit vector $\nabla f(a, b) / |\nabla
	f(a, b)|$ gives the direction in which $f(x, y)$ is increasing most rapidly.
	Furthermore, the maximum rate of increase is $|\nabla f(a, b)|$. Similar
	results hold for $f(x, y, z)$.
\end{theorem}

\begin{facts}
	Tangent planes arise in two situations:
	\begin{itemize}
		\item If $f(x, y)$ is differentiable at $(x_0, y_0)$, then the tangent plane
			to the graph $z = f(x, y)$ at the point $(x_0, y_0, f(x_0, y_0))$ is
			defined by
			\begin{equation}
				z - f(x_0, y_0) = f_x(x_0, y_0) (x - x_0) + f_y(x_0, y_0) (y - y_0).
			\end{equation}
		\item If $F(x, y, z)$ is differentiable at $(x_0, y_0, z_0)$, then $(x_0,
			y_0, z_0)$ lies on the level surface $F(x, y, z) = F(x_0, y_0, z_0)$, and
			the equation of the tangent plane to the surface at this point is defined
			by
			\begin{equation}
				\nabla F(x_0, y_0, z_0) \cdot (x - x_0, y - y_0, z - z_0) = 0,
			\end{equation}
			provided that the gradient $\nabla F(x_0, y_0, z_0)$ is nonzero. Written
			out, this is the equation
			\[
				F_x(x_0, y_0, z_0) (x - x_0) + F_y(x_0, y_0, z_0) (y - y_0) + F_z(x_0,
				y_0, z_0) (z - z_0) = 0.
			\]
			The two situations are related since the graph $z = f(x, y)$ is the level
			surface $F(x, y, z) = f(x, y) - z = 0$. Since $\nabla F = f_x \vec{\imath}
			+ f_y \vec{\jmath} - \vec{k}$, equation (2) reduces to equation (1) in
			this case.
	\end{itemize}
\end{facts}

\section*{Maxima and Minima of Functions of Several Variables}

\begin{definition}
	In two dimensions $(a, b)$ is a critical point of $f(x, y)$ provided
	\[
		f_x(a, b) = f_y(a, b) = 0.
	\]
\end{definition}

\begin{definition}
	A function of two variables has a local maximum at $(a, b)$ if $f(x, y) \leq
	f(a, b)$ when $(x, y)$ is near $(a, b)$. [This means that $f(x, y) \leq f(a,
	b)$ for all points $(x, y)$ in soem disk with center $(a, b)$.] The number
	$f(a, b)$ is called a local maximum value. If $f(x, y) \geq f(a, b)$ when $(x,
	y)$ is near $(a, b)$, then $f$ has a local minimum at $(a, b)$ and $f(a, b)$
	is a local minimum value.
\end{definition}

\begin{fact}
	If $f$ has a local maximum or minimum at $(a, b)$ and the first-order partial
	derivatives of $f$ exist there, then $(a, b)$ is a critical point of $f$.
\end{fact}

\begin{definition}
	Let $(a, b)$ be a critical point of $f$. $(a, b)$ is called a saddle point of
	$f$ if $f(a, b)$ is not a local maximum or minimum.
\end{definition}

\begin{fact}
	For a suitably nice function $f(x, y)$, the second derivative test goes as
	follows. Let $(a, b)$ be a critical point of $f$, and define
	\[
		D = \det
		\begin{pmatrix}
			f_{xx} & f_{xy} \\
			f_{xy} & f_{yy}
		\end{pmatrix}.
	\]
	Then:
	\begin{itemize}
		\item If $D(a, b) > 0$ and $f_{xx}(a, b) > 0$, then $f$ has a local minimum
			at $(a, b)$.
		\item If $D(a, b) > 0$ and $f_{xx}(a, b) < 0$, then $f$ has a local maximum
			at $(a, b)$.
		\item If $D(a, b) < 0$, then $f$ has a saddle point at $(a, b)$.
	\end{itemize}
	The second derivative test is inconclusive in all other cases.
\end{fact}

\begin{fact}
	In a constrained optimization problem, you want to find the maximum or minimum
	of a function subject to a constraint. Such problems occur in two and three
	dimensions and use the method of Lagrange multipliers. We assume that the
	function and constraint are differentiable.
	\begin{itemize}
		\item To maximize or minimize $f(x, y)$ subject to the constraint $g(x, y) =
			0$, solve
			\[
				\nabla f(x, y) = \lambda \nabla g(x, y), \quad g(x, y) = 0,
			\]
			or equivalently,
			\[
				f_x(x, y) = \lambda g_x(x, y), \quad f_y(x, y) = \lambda g_y(x, y),
				\quad g(x, y) = 0.
			\]
		\item To maximize or minimize $f(x, y, z)$ subject to the constraint $g(x, y,
			z) = 0$, solve
			\[
				\nabla f(x, y, z) = \lambda \nabla g(x, y, z), \quad g(x, y, z) = 0,
			\]
			or equivalently,
			\[
				f_x(x, y, z) = \lambda g_x(x, y, z), \quad f_y(x, y, z) = \lambda g_y(x,
				y, z), \quad f_z(x, y, z) = \lambda g_z(x, y, z), \quad g(x, y, z) = 0.
			\]
	\end{itemize}
	It is customary to call $\lambda$ the Lagrange multiplier.
\end{fact}

\begin{fact}
	A problem may ask for the maximum and minimum values (also called extreme
	values) of a differentiable function $f(x, y)$ on a closed and bounded region
	in the plane. Extreme values are known to exist in this situation. They can
	occur in one of two places:
	\begin{itemize}
		\item In the interior of the region, where they occur among the critical
			points of $f$.
		\item On the boundary of th region, where you use Lagrange multipliers. The
			constraint is the defining equation of the boundary.
	\end{itemize}
	Note that when you find the critical points in the interior, you do \emph{not}
	need to apply the second derivative test (which would only tell you about
	local maxima or minima).
\end{fact}

\section*{Double Integrals}

\begin{definition}
	When the region $R$ has a nice description in Cartesian coordinates, the
	double integral can be expressed as an iterated integral in two ways:
	\begin{itemize}
		\item The first way is
			\[
				\iint_R f(x, y)\, dA = \int_a^b \int_{g_1(x)}^{g_2(x)} f(x, y)\,
				dy\, dx
			\]
			when $R$ consists of all points $(x, y)$ where $a \leq x \leq b$, and for
			$x$ in this range, $g_1(x) \leq y \leq g_2(x)$. So $y = g_2(x)$ is the top
			of $R$, $y = g_1(x)$ is the bottom, and $x = a$, $x = b$ are the sides.
			When doing the inner integral $\int_{g_1(x)}^{g_2(x)} f(x, y)\, dy$, you
			should treat $x$ as a constant.
		\item The second way is
			\[
				\iint_R f(x, y)\, dA = \int_c^d \int_{h_1(y)}^{h_2(y)} f(x, y)\, dx\, dy
			\]
			when $R$ consists of all points $(x, y)$ where $c \leq y \leq d$, and for
			$y$ in this range, $h_1(y) \leq x \leq h_2(y)$. From the point of view of
			someone on the $y$-axis, $x = h_2(y)$ is the ``top'' of $R$, $x = h_1(y)$
			is the ``bottom'', and $y = c$, $y = d$ are the ``sides''. When doing the
			inner integral $\int_{h_1(y)}^{h_2(y)} f(x, y)\, dx$, treat $y$ as a
			constant.
	\end{itemize}
\end{definition}

\begin{fact}
	The polar coordinates $(r, \theta)$ of a point are related to the Cartesian
	coordinates $(x, y)$ by the equations
	\[
		r^2 = x^2 + y^2 \quad x = r \cos \theta \quad y = r \sin \theta.
	\]
\end{fact}

\begin{definition}
	When the region $R$ in a double integral $\iint_R f(x, y)\, dA$ has a nice
	description in polar coordinates, the integral can be expressed as the
	iterated integral
	\[
		\iint_R f(x, y)\, dA = \int_\alpha^\beta \int_{h_1(\theta)}^{h_2(\theta)}
		f(r, \theta)\, r\, dr\, d\theta,
	\]
	where $R$ consists of all points with polar coordinates $(r, \theta)$ such
	that $\alpha \leq \theta \leq \beta$, and for $\theta$ in this range,
	$h_1(\theta) \leq r \leq h_2(\theta)$. When doing the inner integral
	$\int_{h_1(\theta)}^{h_2(\theta)} f(r, \theta)\, r\, dr$, you should treat
	$\theta$ as a constant.
\end{definition}

\begin{definition}
	The basic area interpretation of the double integral is $\iint_R 1\, dA =
	\text{Area}(R)$.

	For volumes, there are two situations to consider:
	\begin{itemize}
		\item When $f(x, y) \geq 0$ on $R$, $\iint_R f(x, y)\, dA$ is the volume
			under the surface $z = f(x, y)$ for $(x, y) \in R$.
		\item More generally, suppose that a 3-dimensional region $V$ in $\R^3$
			consists of all points $(x, y, z)$ such that $(x, y) \in R$ and $f_1(x, y)
			\leq z \leq f_2(x, y)$. Thus $z = f_2(x, y)$ is the top of $V$, $z =
			f_1(x, y)$ is the bottom, and the sides lie over the boundary of $R$. In
			this case, the volume of $V$ is
			\[
				\text{Vol}(V) = \iint_R (f_2(x, y) - f_1(x, y))\, dA.
			\]
	\end{itemize}
\end{definition}

\section*{Triple Integrals}

\begin{definition}
	Given a function $f(x, y, z)$ on a region $R$ in $\R^3$, one can define the
	triple integral $\iiint_R f(x, y, z)\, dV$.
	\begin{itemize}
		\item Cartesian coordinates $x, y, z$, where $dV = dx\, dy\, dz$ or $dz\,
			dy\, dx$. Other orders are possible.
		\item  Cylindrical coordinates $r, \theta, z$, where $dV = r\, dz\, dr\,
			d\theta$.
			\begin{itemize}
				\item To convert from cylindrical to Cartesian coordinates, we use these
					equations
					\[
						x = r \cos\theta \quad y = r \sin\theta \quad z = z.
					\]
				\item To convert from Cartesian to cylindrical coordinates, we use these
					equations
					\[
						r^2 = x^2 + y^2 \quad \tan\theta = \frac{y}{x} \quad z = z.
					\]
			\end{itemize}
		\item Spherical coordinates $\rho, \phi, \theta$, where $dV = \rho^2\,
			\sin\phi\, d\rho\, d\phi\, d\theta$.
			\begin{itemize}
				\item To convert from spherical to Cartesian coordinates, we use these
					equations
					\[
						x = \rho \sin\phi \cos\theta \quad y = \rho \sin\phi \sin\theta
						\quad z = \rho \cos\phi.
					\]
				\item To convert from Cartesian to spherical coordinates, we use the
					equation
					\[
						\rho^2 = x^2 + y^2 + z^2.
					\]
			\end{itemize}
	\end{itemize}
\end{definition}

\begin{definition}
	The basic volume interpretation of the triple integral is $\iiint_R 1\, dV =
	\text{Vol}(R)$.
\end{definition}
